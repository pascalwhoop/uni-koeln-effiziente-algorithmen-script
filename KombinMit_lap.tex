\documentclass[12pt,titlepage,a4paper] {report}
\usepackage[latin1]{inputenc}
\usepackage{ngerman}

%\usepackage[cspex,bbgreekl]{mathbbol}
\usepackage{amsmath,amssymb,latexsym}
\usepackage{bbm}
\usepackage{amsmath}

\usepackage{pslatex}
\usepackage{graphicx}
\usepackage[bottom,marginal]{footmisc}
\usepackage{setspace}
\usepackage{vmargin}

%\setpapersize{USletter}
%\setmargins{2.8cm}{1.1cm}% % linker & oberer Rand
%             {16cm}{23.5cm}%   % Textbreite und -hoehe
%             {12pt}{25pt}%   % Kopfzeilenhoehe und -abstand
%             {0pt}{30pt}%    % \footheight (egal) und Fusszeilenabstand

%\usepackage{atbeginend}

%\AfterBegin{itemize}{\setlength{\itemsep}{-2pt} }


%\renewcommand{\familydefault}{ppl}

\parskip5pt plus3pt
\parindent0pt
%\doublespace      
\linespread{1.2}
%\onehalfspacing  
\title{Polynomielle Kombinatorische Optimierungsalgorithmen}
\author{Hartmut Wahl}
%\setcounter{tocdepth}{4}
%\setcounter{secnumdepth}{4}
\usepackage{textcomp}


\def\abstractname{Hinweise}
\newcommand{\RR}{\mathbb{R}}
\newcommand{\NN}{\mathbb{N}}
\newcommand{\ZZ}{\mathbb{Z}}


%\selectlanguage{USenglish}
\begin{document}
\begin{titlepage}
\begin{singlespacing}
\begin{tabular}{l}
{\Large Universität zu Köln}\\

\vspace{9cm}\\


{\bf \large Polynomielle Kombinatorische Optimierungsalgorithmen }\\
Mitschrift der Vorlesung von Prof. M. Jünger\\
WS 2002/2003
\\
\vspace{9cm}
\\
Hartmut Wahl\\
hwahl@hwahl.de\\
\vspace{7mm}\\

\end{tabular}
\end{singlespacing}
\end{titlepage}

\pagenumbering{Roman}

\begin{abstract}
Diese Mitschrift wurde von mir angefertigt, um mir das Lernen des Stoffes zu
erleichtern und ist keine offizielle Mitschrift des Lehrstuhls. Sie ist
nicht fehlerkorrigiert und sollte mit einer gehörigen Portion Misstrauen
genossen werden. Korrekturen und Anregungen bitte an mich.\\
\\
Ich übernehme keine Garantie für Richtigkeit: Wer eine Garantie will kaufe
sich einen Toaster!\\
\\
Literatur:\\
Prof. Jünger wies auf dem Semesterapparat hin, der sich in der Bibliothek
der Informatik im Polighaus befindet in dem sich ein Buch befindet:\\
{\bf Willam J. Cook, William H. Cunningham, William R. Pulleyblank, 
Alexander Schrijver} "`Combinatorial Optimization"', New York: Wiley 1998.\\
Dieses Buch umfasst einen Großteil des Stoffes der Vorlesung.

\end{abstract}


\begin{onehalfspacing}
\tableofcontents
\end{onehalfspacing}
\pagebreak
%\listoftables
\pagenumbering{arabic}

%{\sffamily\texteuro}


\setcounter{chapter}{-1}

\chapter{Einführung}

\newtheorem{satz}{Satz}[chapter]
\newtheorem{lemma}{Lemma}[chapter]

Vor der Einführung machte Prof. Jünger eine ganz kurze informelle 
Vorstellung der kombinatorischen Optimierung, bei der er uns bat nicht 
mit zuschreiben. Dieser Teil ist deshalb hier nicht wieder gegeben. 

\section{Allgmeine Formulierungen von Optimierungsproblemen}

\subsection{Generisches Optimierungsproblem}
\begin{tabular}{ll}
Gegeben: & $ A \in \RR^{p \times r},\;  B \in \RR^{p\times r},\; 
C \in \RR^{q\times r}, \; D \in \RR^{q\times s} $\\
& $a \in \RR^{r}, \; b \in \RR^{s}, \; c \in \RR^{p},\; d \in \RR^{q}$\\
(LP)&$\max a^{T}x + b^{T}y$\\
&$Ax+ By = c$\\
&$Cx+Dy \leq d$\\
&$x \geq 0$\\
\end{tabular}\\

Dies ist die allgemeine Formulierung eines (linearen) generischen
Optimierungsproblems. Für ein ganzzahliges Optimierungsproblem (IP) gilt
zusätzlich zum LP noch:

\[x \in \ZZ^{r}, \; \in \ZZ^{s} \]

Ein gemischt ganzzahliges Optimierungsproblem liegt vor wenn nur Teile
der Vektoren x und y integral (ganzzahlig) sind.


\subsection{Kombinatorisches Optimierungsproblem}

\begin{tabular}{ll}
Gegeben:& $E$ eine endliche Grundmenge\\
&$y \subseteqq 2^{E},\;$ y ist eine zulässige Teilmenge von $E$\\
&$c$: $E \rightarrow \RR$ (z.b. Zuweisung von Entfernungen zu den
Kanten)\\
&Für $F \subseteqq E$: $c(F) = \sum_{e\in F} c(e)$\\
Gesucht:& $I^{\star} \in Y$ mit $c(I^{\star} \geqq c(I),\;$ für alle $I \in
Y$\\
\hspace{3mm} oder:&$I^{\star} \in Y$ mit $c(I^{\star} \leqq c(I),\;$ für
alle $I \in Y$
\end{tabular}


\subsubsection{Kleine Aufwärmaufgabe}

Für $A \in \RR^{m\times n}, \; b \in \RR^{m}$ gilt:\\
($\exists x \in \RR^{n}$) $\;Ax = b$ (Gleichungssystem ist lösbar)\\
genau dann wenn:\\
($\nexists y \in \RR^{m}$) $\;y^{T}A=0$, $\;y^{T} b=-1$

Hiernach erklärte Prof. Jünger noch kurz und informell die Triangulation
für Input-/Outputmodelle. In der Nächsten Stunde wurden zunächst 
Reihenfolgeprobleme mit einigen Beispielen im Graphenzeichnen und in der 
Flugplanung veranschaulicht.

\section{Einführung in die lineare Optimierung}

Es folgt ein Beispiel aus der Ölraffinierung: Beim Prozess des so genannten
"`Crackens"' wird Rohöl in drei Produkte aufgespalten: Schweres Öl (S),
mittelschweres Öl (M) und leichtes Öl (L). Nun gibt es zwei verschiedene
Crack-Prozesse die zu einer unterschiedlichen Aufteilung der Endprodukte
führen.

\[\mbox{Prozess 1: 10 ME (= Mengeneinheiten) Rohöl ergibt} \longrightarrow 
\left. \begin{array}{l}
\mbox{2 ME S}\\
\mbox{2 ME M}\\
\mbox{1 ME L}\\
\end{array}\right\} \mbox{Kosten 3 GE}\]

\[\mbox{Prozess 1: 10 ME (= Mengeneinheiten) Rohöl ergibt} \longrightarrow
\left.
\begin{array}{l}
\mbox{1 ME S}\\
\mbox{2 ME M}\\
\mbox{4 ME L}\\
\end{array}\right\}\mbox{Kosten 5 GE}\]

Die Firma hat Lieferverpflichtungen, deswegen muss sie mindestens 3 ME S, 5
ME M und 4 ME L produzieren.

\[\begin{array}{ll}
\mbox{Variablen:}& x_{1}\mbox{, Produktionsniveau von Crackprozess 1}\\
&x_{2}\mbox{, Produktionsniveau von Crackprozess 2}\\
&x_{1}\geqq 0, x_{2}\geqq 0\\
&\mbox{z.B. }x_{1} = 1,5 \Leftrightarrow \mbox{ 15 ME Rohöl für Prozess 1}
\end{array}\]

Damit erhalten wie das folgende lineare Optimierungsproblem:\\
\[\begin{array}{l}
\min\; 3x_{1}+5x_{2} \;\;\; \mbox{(Zielfunktion)}\\
\left.\begin{array}{l}\mbox{s.t.}\begin{array}[t]{rl}
(1)& 2x_{1} + x_{2} \geqq 3\\
(2)&2x_{1}+ x_{2} \geqq 5\\
(3)&x_{1}+ 4x_{2} \geqq 4\\
(4)&x_{1} \geqq 0\\
(5)&x_{2} \geqq 0\\
\end{array}\end{array}\right\}\mbox{Restriktionen}\end{array}
\]

Jeder Vektor ${\scriptsize\left(\begin{array}{c}x_{1}\\x_{2}\end{array}\right)} \in \RR^{2}$, der
(1), (2)...., (5) erfüllt heißt zulässige Lösung.

Graphische Darstellung (grad keine Lust auf xfig, das bekommt ihr ja wohl
selber hin :)

Aus der n.v. Graphik kann man ganz einfach die optimale Lösung erkennen:
$x^{*} = {\scriptsize\left(\begin{array}{c}2\\0,5\end{array}\right)}$, Wert = 8,5.

Nähme man nun aber eine beliebige Zielfunktion $ax_{1}+bx_{2}$ sowie $a < 0$ 
und definiere eine Folge von Punkten $x^{(n)}= {\scriptsize\left(
\begin{array}{c}n\\ 0\end{array}\right)}$ mit $n \in \NN$. Weiterhin gilt: 
$ax_{1}^{(n)} + bx_{2}^{(n)} = na$. Da a negativ 
ist ist der Wert der Zielfunktion offensichtlich nicht nach unten
beschränkt.

\subsection{Lösungen für Lineare Probleme}

\subsubsection{Simpler Lösungsalgorithmus für ein LP}  

Zunächst stellen wir eine Vermutung an. Wenn eine Optimallösung
existiert, so existiert eine die im Durchschnitt von n Hyperebenen liegt
(wird später bewiesen). Eine Hyperebene ist ein (n-1)-dimensionaler
Unterraum also im $R^{2}$ eine Gerade im $R^{3}$ eine Ebene, u.s.w.

Davon ausgehend konstruieren wie eine einfache Lösungsmethode. Man
schreibe (1), (2), ..., (5) als Gleichungen und löse alle ${5\choose 2}
= 10$ Gleichungen mit 2 der Gleichungen. Von allen Lösungen, die
zulässig sind, ist die mit dem kleinsten Zielfunktionswert optimal.

{\bf Aufwand:} Benötigen ${m\choose n}$ Aufrufe des Gleichungssytemlösers mit
n Variablen und n Gleichungen. $\rightarrow$ Viel zu langsam in der
Praxis.

\subsubsection{Idee des Simplexalgorithmus}

Der Simplexalgorithmus geht folgendermaßen vor:\\
(a) Finde eine erste Lösung wie im simplen Algorithmus.\\
(b) Gehe durch Austausch je einer Hyperebene zu einer benachbarten
Lösung deren Zielfunktionswertes nicht schlechter ist.
(c) Wiederhole (b) bis Optimalität festgestellt ist. ($\rightarrow$ via
Qualitätstheorie, die später noch behandelt wird).

\subsubsection{Bestimmung von Schranken}

Wir zeigen nun, dass $x^{*} =
{\scriptsize\left(\begin{array}{c}2\\0,5\end{array}\right)}$ optimal
ist:\\
{\bf Beobachtung 1:} Die Lösungsmenge einer Ungleichung ist invariant
unter Skalierung mit positiven Skalaren, z.B. (1):
\[\{x|2x_{1}+x_{2} \geqq 3\} = \{x|4x_{1}+2x_{2} \geqq 6\}\]

Negative Faktoren kehren die Ungleichung um.

{\bf Beobachtung 2:} Erfüllt ein Vektor zwei gleich gerichtete (beide
"`$\leqq$"' oder beide "`$\geqq$"') so auch die Summe der beiden
Ungleichungen. z.B. (1) + (3):
\[\begin{array}{llcl}
(1)&2x_{1}+x_{2} &\geqq& 3\\
(3)&x_{1}+4x_{2}&\geqq&4\\
\hline
&\underbrace{3x_{1} + 5x_{2}}_{ZF}&\geqq& 7
\end{array}\] 

Nun können wir behaupten, dass wir mindestens 7 GE ausgeben müssen.
$\rightarrow$ Jeder Vektor der (1)...(5) erfüllt hat einen ZF-Wert von
wenigstens 7. Also ist 7 eine untere Schranke für das Minimum. Nun versuchen
wir eine bessere schranke zu finden:
\[\begin{array}{llcl}
(1)&2x_{1}+x_{2} &\geqq& 3\\
(2)&2x_{1}+2x_{2}&\geqq&5\\
\hline
&4x_{1} + 3x_{2}&\geqq& 8
\end{array}\]
In dieser Ungleichung ist der Koeffizient für $x_{1}$ 4 und ist damit
größer als der Koeffizient in der Zielfunktion. Damit ist dies Ungleichung
nicht zum Abschätzen einer unteren Schranke geeignet. Also ein erneuter
Versuch indem wir (2) mit 1,5 multiplizieren. Damit erhalten wir:
\[3x_{1}+3x_{2} \geqq 7,5\]

\subsubsection{Dualität}

Zwar haben wir nun eine bessere untere Schranke aber dieser Probieransatz ist
etwas unbefriedigend. Darum nun ein systematischer Ansatz
zum Finden der Faktoren für die Gleichungen um eine Schranke zu finden. Da
die Nichtnegativbedingen in diesem Fall irrelevant sind suchen wir für
unser Problem drei Multiplikatoren ($y_{1}$, $y_{2}$, $y_{3}$) für die
Gleichungen (1) (2) (3).
\[\begin{array}{l}\max \; 3y_{1} + 5y_{2} + 4y_{3} \rightarrow \mbox{Die Rechte
Seite (Schranke) maximieren)}\\
\left.\begin{array}{l}\mbox{s.t.}\begin{array}[t]{lcl}
2y_{1}+2y_{2}+3y_{3} &\leqq & 3 \\
y_{1}+2y_{2}+4y_{3} &\leqq & 5
\end{array} \end{array} \right\}\mbox{Damit die ZF-Koeffizienten nicht
überschritten werden}\end{array}\]
Damit haben wir nun das entsprechende duale Optimierungsproblem. Nach
dessen Konstruktion liefert der Zielfunktionswert jeder zulässigen Lösung
des dualen Problems eine untere Schranke für den Zielfunktionswer des
primalen Problems. Wir betrachten:
\[y^{*}=\left(\begin{array}{c}y^{*}_{1}\\y^{*}_{2}\\y^{*}_{3}\end{array}\right)
=\left(\begin{array}{c}0\\\frac{7}{6}\\\frac{2}{3}\end{array}\right)\] 
dies entspricht:
\[\begin{array}{llcl}
\frac{7}{6} \cdot (2):\; & \frac{14}{6}x_{1} + \frac{14}{6}x_{2} &\geqq&
\frac{35}{6}\\
\frac{2}{3} \cdot (3):\; & \frac{4}{6}x_{1} + \frac{16}{6}x_{2} &\geqq&
\frac{16}{6}\\
\hline
&3x_{1} + 5x_{2} &\geqq& 8,5
\end{array}
\]

\begin{satz}
Schwacher Dualitätssatz:\\
Es seien $c \in \RR^{n}$, $b \in \RR^{m}$, $A \in \RR^{m \times n}$ für die
Optimierungsaufgaben:

\begin{tabular}{ll}
(P)&$\min \{c^{T}x | A \geqq b, \; x\geqq 0\}$\\
&d.h. bestimme $x^{*} \in \RR$ mit $Ax^{*} \geqq b$, $x^{*} \geqq 0 $ und
$c{T}x^{*}$ so klein wie möglich.\\
(D)&$\max \{y^{T}b|y^{T}A \leqq c^{T}, \; y \geqq 0\}$\\
&d.h. bestimme  $y^{*} \in \RR^{m}$ mit $g^{*T}A \leqq c^{T}, \; y ^{*}
\geqq 0$ und $y^{*T}b$ so groß wie möglich.
\end{tabular}

so gilt folgendes: Seien $x_{0} \in \RR^{n} und y_{0} \in \RR^{m}$ Punkte
mit $Ax_{0} \geqq b $, $x_{0} \geqq 0$ und $y_{0}^{T}A \leqq c^{T}$,
$y_{0} \geqq 0$ dann gilt $y_{0}^{T}b \leqq c^{T} x_{0}$\\
Beweis:
\[y_{0}^{T}b \leqq y_{0}^{T}(Ax_{0}) = (y_{0}^{T}A)x_{0} \leqq c^{T}x_{0}
\; \; \; \; \mbox{q.e.d.}\] 
\end{satz}

Der starke Dualitätssatz (später) besagt $y^{*T}b =c^{T}x^{*}$ für
Optimallösung $x^{*}$ von (P) und $y^{*}$ von (D).

\subsubsection{Ökonomische Interpretation des Dualen Programms}

Ein Manager der Ölraffinie würde sich nun die Frage stellen, ob es sinnvoll
L, M uns S selbst herzustellen, oder ob man es auf dem Markt kaufen solte 
b.z.w.  bei welchen Marktpreis eine Produktion nicht mehr lonht. In diesem Fall
drückt $y_{1}$ den Preis in GE für S aus, $y_{2}$ für M, $y_{3}$ für L. Der
Marktpreis für die benötignen Mengen beträgt somit:
\[3y_{1} + 5y_{5} + 4y_{3} \; \; \mbox{GE}\]
der Ausstoß von Crackprozess 1 hat den Marktpreis:
\[2y_{1}+2y_{2}+3y_{3} \; \; \mbox{GE} > 3 \;\;\;\; \rightarrow \mbox{Produktion Lohnt}\]
der Ausstoß von Crackprozess 2 hat den Marktpreis:
\[y_{1}+2y_{2}+4y_{3} \;\; \mbox{GE} > 5 \;\;\;\; \rightarrow \mbox{Produktion Lohnt}\]
$y_{i}$ werden auch als Schattenpreise bezeichnet. In unserem Beispiel
betragen sie:
\[\begin{array}{lll}(S):\; y_{1}^{*}=0& (M):\; y_{2}^{*} =
\frac{7}{6}&(L):\;
y_{3}^{*} = \frac{2}{3}
\end{array}\]
Hier ist zu erkennen dass S praktisch "`wertlos"' für die Firma ist. In der
tat ist es in der Optimallösung im Überfluss vorhanden (also mehr als die
Lieferverpflichtungen der Firma verlangen. D.h. man würde das Öl zunächst
zu jedem Preis verkaufen (solange man die insgesamte Produktionsmenge
dadurch nicht erhöhen muss).

\subsection{Formen von Linearen Problemen}

\subsubsection{Kanonische Form eines LPs}

Die Kanonische Form eines LPs lautet:
\[\begin{array}{lcl}
\max\; c^{T}x\\
Ax&\leqq& b\\
x&\geqq& 0
\end{array}\]
Transformationen von anderen Standardformen in die kanonische Form:
\[\begin{array}{l@{\hspace{3mm}}rcl@{\hspace{8mm}}c@{\hspace{8mm}}rcl}
1)&\max  a^{T}y&&&&\max a^{T}x\\
&By&=&d&\stackrel{x=y}{\longrightarrow}&{\scriptsize\left(\begin{array}{r}B\\-B\\\end{array}
\right)}x&\leqq&{\scriptsize\left(\begin{array}{r}d\\-d\end{array}
\right)}\\
&y&\geqq& 0&&x&\geqq& 0\vspace{5mm}\\
2)&\max a^{T}y&&&&\max {\scriptsize\left(\begin{array}{r}a\\-a\end{array}\right)}^{T}x\vspace{2mm}\\
&By&\leqq&d&\stackrel{x={\scriptsize\left(\begin{array}{r}y+\\y-\end{array}
\right)}}{\longrightarrow}&(B, -B)x&\leqq& d\\
&&&&&x&\geqq& 0\vspace{5mm}\\
3)&\max a^{T}y&&&&\max
{\scriptsize\left(\begin{array}{r}a\\-a\end{array}\right)}^{T}x\vspace{2mm}\\
&By&=&d&\stackrel{x={\scriptsize\left(\begin{array}{r}y+\\y-\end{array}
\right)}}{\longrightarrow}&   
{\scriptsize\left(\begin{array}{rr}B&-B\\-B&B\end{array}\right)}
x&\leqq& {\scriptsize\left(\begin{array}{r}d\\-d\end{array}
\right)}\\
&&&&&x&\geqq& 0
\end{array}\]
Als Hausaufgabe dürfen wir die Transformationen in die andere Richtung
machen :), (bei $By=d$ geht es nicht).

Jedes LP kann also auch in die Form $\max \{c^{T}| Ax \leqq b\}$ transformiert
weden.

\begin{satz} Farkas Lemma\\
Wir suchen ein Analogon zur "`Aufwärmaufgabe"':\\
$A \in \RR^{m\times n}, \; \; b \in \RR^{m}$\\
$(\exists x) \; \;\; Ax = b$ genau dann wenn $(\not\exists y) \; \; y^{T}A
= 0, \; \; y^{T}b=-1$ für Ungleichungsprobleme der Form $Ax\leqq b$

\end{satz}

\paragraph{Fourier-Motzkin Elimination}

Durch Multiplikation der Ungleichungen mit positiven Skalaren und Umordnung
erhalten wir:
\[
(*)\left\{\begin{array}{rcl}
x_{1} + (a_{i}')^{T}  x'&\leqq& b^{*}_{i} \;\; \forall \; i\; \in \{1,2,3,...m'\}\;\;\; \mbox{aus allen wird 1}\\
-x_{1} + (a_{i}')^{T}  x'&\leqq& b^{*}_{i} \;\; \forall \; i\; \in
\{m'+1,...m''\}\;\;\; \mbox{aus allen wird -1}\\
x_{1} + (a_{i}')^{T}  x'&\leqq& b^{*}_{i} \;\; \forall \; i\; \in
\{m''+1,...m\}\\
\end{array}\right.
\]
\begin{tabular}{ll}
wobei&$x'=(x_{2}, x_{3},...x_{n})$\\
&$(a_{i}')^{T}$ ist die i-te Zeile von A ohne den ersten Eintrag
\end{tabular}

Die ersten beiden Zeilen von (*) sind äquivalent zu
\[\max_{m'+1\leqq j \leqq m''} \left((a_{j}')^{T} x' -b_{j}\right) \leqq x_{i}
\leqq \min_{1\leqq i \leqq m'} \left(b_{i} - (a_{i}')^{T}x'\right)\]

d.h. das Orginalsystem $A \leqq b$ hat eine Lösung genau dann wenn (in
Zukunft g.d.w.) das System:
\[
(**)\left\{\begin{array}{rcl}
(a_{j}')^{T}x' -b_{j} &\leqq& b_{i} - (a_{i}')^{T}x'\;\;\; \forall \; i \; \in
\{1,2,3,...m'\}\\
(a_{j}')^{T}x' -b_{j} &\leqq& b_{i} - (a_{i}')^{T}x'\;\;\; \forall \; i \; \in
\{m'+1,...m''\}\\
(a_{j}')^{T}x' -b_{j} &\leqq& b_{i} - (a_{i}')^{T}x'\;\;\; \forall \; i \; \in
\{m''+1,...m\}\\
\end{array}\right.
\]
eine Lösung hat ($x_{i}$ kann geeignet gewählt werden).\\
Umformulierung von (**)
\[
(***)\left\{\begin{array}{rcl}
(a_{i}'+a_{j}')^{T} x' &\leqq& b_{i} + b_{j}\;\;\; \forall \; i \; \in
\{1,2,3,...m'\}\\
(a_{i}'+a_{j}')^{T} x' &\leqq& b_{i} + b_{j}\;\;\; \forall \; i \; \in
\{m'+1,...m''\}\\
(a_{i}'+a_{j}')^{T} x' &\leqq& b_{i} + b_{j}\;\;\; \forall \; i \; \in
\{m''+1,...m\}\\
\end{array}\right.
\]
Schreibe (***) als $A'x'\leqq b'$\\
Also $(\exists x) \; \; Ax\leqq b$ g.d.w. $(\exists x') A' x' \leqq b'$

Fazit: Wir haben die Variable x eliminiert und haben jetzt in (***) $n-1$
Variablen und $m'(m''-m')+m-m''$ Ungleichungen $\rightarrow$ weniger
Variablen aber mehr Ungleichungen.

Diese Iteration liefert ein nicht polynomielles Verfahren zur Lösung von
Ungleichungssystemen.

\begin{satz}
Lemma von Farkas für Ungleichungen\\
Seien $A \in \RR^{m\times n}$, $b\in \RR^{m}$\\
$(\exists x \in \RR^{n})$ $Ax \leqq b\;$ g.d.w. $\;(\not\exists y \in \RR^{m})$
$y\geqq 0$, $y^{T}A=0$, $y^{T}b < 0$
\end{satz}


%
%  Ab hier weiter gemacht
%

Fall 1 $\mu =0$:
\[\Rightarrow \left\{ \begin{array}{rcl}
Az &\geqq & 0\\
c^{T}z&<&0\\
\end{array}\right.\]
Sei $x^{0}$ Lösung von$Ax\leqq b$. Für $r$ hinreichend groß gilt dann:
\[\begin{array}{rcl}A(x^{0} -rz) &\leqq& b\\
c^{T}(x^{0}-rz)&>&\delta\end{array}\]
$\bar{X}=x^{0} -rz$ ist Lösung die $c^{T}\bar{x} \leqq \delta$ verletzt.

Fall 2 $\mu > 0$:\\
Es gilt $Az + b \mu \geqq 0$
\[\begin{array}{crcl}
\Rightarrow & Az &\geqq & -b \mu\\
\Rightarrow & A\left(-\frac{z}{\mu} \right)&>& \delta
\end{array}\]
aber $c^{T}z + \delta\mu < 0$
\[\begin{array}{crcl}
\Rightarrow & c^{T}z &<& - \delta \mu \\
\Rightarrow & c^{T}\left(-\frac{z}{\mu}\right) &>&
\end{array}\]

Also liefert die Lösung $\bar{x}=-\frac{z}{\mu}$ den Widerspruch.

\section{Qualitätstheorem}

Für das lineare Optimierugsproblem:
\[(P) ;\; \; \{\max c^{T}x | Ax \leqq b\}\]
bezeichnen wir:
\begin{itemize}
\item Jede Lösung $\bar{x}$ von $Ax \leqq b$ als zulässige Lösung
\item Jede zuläassige Lösung $\bar{x} die c^{T} x $ maximiert als
Optimallösung. 
\end{itemize}
Wir definieren das duale lineare Optimierungsproblem zu (P)
als (D) \[\min \{y^{T}b| y \leqq 0 , \; \; y^{T}A = c^{T} \}\]

\begin{satz} \label{SchwQual} Schwaches Qualitätstheorem\\
Seien $A \in \RR^{m \times n}, \; \; b \in \RR^{m}, \;\; c \in \RR^{n}$ und
$\bar{x}$ zulässig  für (P) und y zulässig für (D). Dann gilt $c^{T}\bar{x}
\leqq \bar{y}^{T} b$
\end{satz}
Beweis: $c^ {T}\bar{x} = (y^{T}A) \bar{X} = y^{T}(Ax) \leqq y^{T}b$ q.e.d.

\begin{satz} 
Qualitätstheorem (von Neumann, 1947)\\
Seien $A \in \RR^{m \times n}, \; \;  b \in \RR^{m}, \;\; c \in \RR^{n}$.
Dann gilt:
\[\max c^{T}x | Ax \leqq b\}=\min \{y^{T}b| y \leqq 0 , \; \; y^{T}A = 
c^{T} \}\]
\end{satz}
Beweis:
\begin{enumerate}
\item Schwache Dualität (Satz \ref{SchwQual}) implizerit:
\[\begin{array}{clcl}\]


\end{enumerate}



%\chapter{Probleme, Schranken, Zetifikate}


%\chapter{Optimale Bäume und Wege}
%\chapter{Maximale Flüsse}
%\chapter{Minimale Kostenflüsse}
%\chapter{Optimale Matchingsysteme}




%\small
%\begin{thebibliography}{99}
%\input{ISO14000_bib}
%\end{thebibliography}



\end{document}










